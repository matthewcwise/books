\chapter{Preface}
\normalsize
\vspace{-1em}
\footnotesize{
    \begin{flushright}
    \textit{June 13, 2025}
    \end{flushright}
\vspace{-1em}
Dear Greta,

Happy Birthday! I'm so grateful to keep continuing our life's adventures together.

I'm eternally grateful for you, the love we share

\begin{flushright}
    \textemdash{} Love, Matthew
\end{flushright}
}



\chapter{Additional Preface}

\emph{[Illustration]}

\textit{Walt Whitman has somewhere a fine and just distinction between ``loving by allowance'' and ``loving with personal love.'' This distinction applies to books as well as to men and women; and in the case of the not very numerous authors who are the objects of the personal affection, it brings a curious consequence with it. There is much more difference as to their best work than in the case of those others who are loved ``by allowance'' by convention, and because it is felt to be the right and proper thing to love them. And in the sect---fairly large and yet unusually choice---of Austenians or Janites, there would probably be found partisans of the claim to primacy of almost every one of the novels. To some the delightful freshness and humour of} Northanger Abbey, \textit{its completeness, finish, and} entrain, \textit{obscure the undoubted critical facts that its scale is small, and its scheme, after all, that of burlesque or parody, a kind in which the first rank is reached with difficulty.} Persuasion, \textit{relatively faint in tone, and not enthralling in interest, has devotees who exalt above all the others its exquisite delicacy and keeping. The catastrophe of} Mansfield Park \textit{is admittedly theatrical, the hero and heroine are insipid, and the author has almost wickedly destroyed all romantic interest by expressly admitting that Edmund only took Fanny because Mary shocked him, and that Fanny might very likely have taken Crawford if he had been a little more assiduous; yet the matchless rehearsal-scenes and the characters of Mrs. Norris and others have secured, I believe, a considerable party for it.} Sense and Sensibility \textit{has perhaps the fewest out-and-out admirers; but it does not want them.}

\textit{I suppose, however, that the majority of at least competent votes would, all things considered, be divided between} Emma \textit{and the present book; and perhaps the vulgar verdict (if indeed a fondness for Miss Austen be not of itself a patent of exemption from any possible charge of vulgarity) would go for} Emma. \textit{It is the larger, the more varied, the more popular; the author had by the time of its composition seen rather more of the world, and had improved her general, though not her most peculiar and characteristic dialogue; such figures as Miss Bates, as the Eltons, cannot but unite the suffrages of everybody. On the other hand, I, for my part, declare for} Pride and Prejudice \textit{unhesitatingly. It seems to me the most perfect, the most characteristic, the most eminently quintessential of its author's works; and for this contention in such narrow space as is permitted to me, I propose here to show cause.}

\textit{In the first place, the book (it may be barely necessary to remind the reader) was in its first shape written very early, somewhere about 1796, when Miss Austen was barely twenty-one; though it was revised and finished at Chawton some fifteen years later, and was not published till 1813, only four years before her death. I do not know whether, in this combination of the fresh and vigorous projection of youth, and the critical revision of middle life, there may be traced the distinct superiority in point of construction, which, as it seems to me, it possesses over all the others. The plot, though not elaborate, is almost regular enough for Fielding; hardly a character, hardly an incident could be retrenched without loss to the story. The elopement of Lydia and Wickham is not, like that of Crawford and Mrs. Rushworth, a} coup de théâtre; \textit{it connects itself in the strictest way with the course of the story earlier, and brings about the denouement with complete propriety. All the minor passages---the loves of Jane and Bingley, the advent of Mr. Collins, the visit to Hunsford, the Derbyshire tour---fit in after the same unostentatious, but masterly fashion. There is no attempt at the hide-and-seek, in-and-out business, which in the transactions between Frank Churchill and Jane Fairfax contributes no doubt a good deal to the intrigue of} Emma, \textit{but contributes it in a fashion which I do not think the best feature of that otherwise admirable book. Although Miss Austen always liked something of the misunderstanding kind, which afforded her opportunities for the display of the peculiar and incomparable talent to be noticed presently, she has been satisfied here with the perfectly natural occasions provided by the false account of Darcy's conduct given by Wickham, and by the awkwardness (arising with equal naturalness) from the gradual transformation of Elizabeth's own feelings from positive aversion to actual love. I do not know whether the all-grasping hand of the playwright has ever been laid upon} Pride and Prejudice; \textit{and I dare say that, if it were, the situations would prove not startling or garish enough for the footlights, the character-scheme too subtle and delicate for pit and gallery. But if the attempt were made, it would certainly not be hampered by any of those loosenesses of construction, which, sometimes disguised by the conveniences of which the novelist can avail himself, appear at once on the stage.}

\textit{I think, however, though the thought will doubtless seem heretical to more than one school of critics, that construction is not the highest merit, the choicest gift, of the novelist. It sets off his other gifts and graces most advantageously to the critical eye; and the want of it will sometimes mar those graces---appreciably, though not quite consciously---to eyes by no means ultra-critical. But a very badly-built novel which excelled in pathetic or humorous character, or which displayed consummate command of dialogue---perhaps the rarest of all faculties---would be an infinitely better thing than a faultless plot acted and told by puppets with pebbles in their mouths. And despite the ability which Miss Austen has shown in working out the story, I for one should put} Pride and Prejudice \textit{far lower if it did not contain what seem to me the very masterpieces of Miss Austen's humour and of her faculty of character-creation---masterpieces who may indeed admit John Thorpe, the Eltons, Mrs. Norris, and one or two others to their company, but who, in one instance certainly, and perhaps in others, are still superior to them.}

\textit{The characteristics of Miss Austen's humour are so subtle and delicate that they are, perhaps, at all times easier to apprehend than to express, and at any particular time likely to be differently apprehended by different persons. To me this humour seems to possess a greater affinity, on the whole, to that of Addison than to any other of the numerous species of this great British genus. The differences of scheme, of time, of subject, of literary convention, are, of course, obvious enough; the difference of sex does not, perhaps, count for much, for there was a distinctly feminine element in ``Mr. Spectator,'' and in Jane Austen's genius there was, though nothing mannish, much that was masculine. But the likeness of quality consists in a great number of common subdivisions of quality---demureness, extreme minuteness of touch, avoidance of loud tones and glaring effects. Also there is in both a certain not inhuman or unamiable cruelty. It is the custom with those who judge grossly to contrast the good nature of Addison with the savagery of Swift, the mildness of Miss Austen with the boisterousness of Fielding and Smollett, even with the ferocious practical jokes that her immediate predecessor, Miss Burney, allowed without very much protest. Yet, both in Mr. Addison and in Miss Austen there is, though a restrained and well-mannered, an insatiable and ruthless delight in roasting and cutting up a fool. A man in the early eighteenth century, of course, could push this taste further than a lady in the early nineteenth; and no doubt Miss Austen's principles, as well as her heart, would have shrunk from such things as the letter from the unfortunate husband in the} Spectator, \textit{who describes, with all the gusto and all the innocence in the world, how his wife and his friend induce him to play at blind-man's-buff. But another} Spectator \textit{letter---that of the damsel of fourteen who wishes to marry Mr. Shapely, and assures her selected Mentor that ``he admires your} Spectators \textit{mightily''---might have been written by a rather more ladylike and intelligent Lydia Bennet in the days of Lydia's great-grandmother; while, on the other hand, some (I think unreasonably) have found ``cynicism'' in touches of Miss Austen's own, such as her satire of Mrs. Musgrove's self-deceiving regrets over her son. But this word ``cynical'' is one of the most misused in the English language, especially when, by a glaring and gratuitous falsification of its original sense, it is applied, not to rough and snarling invective, but to gentle and oblique satire. If cynicism means the perception of ``the other side,'' the sense of ``the accepted hells beneath,'' the consciousness that motives are nearly always mixed, and that to seem is not identical with to be---if this be cynicism, then every man and woman who is not a fool, who does not care to live in a fool's paradise, who has knowledge of nature and the world and life, is a cynic. And in that sense Miss Austen certainly was one. She may even have been one in the further sense that, like her own Mr. Bennet, she took an epicurean delight in dissecting, in displaying, in setting at work her fools and her mean persons. I think she did take this delight, and I do not think at all the worse of her for it as a woman, while she was immensely the better for it as an artist.}

\textit{In respect of her art generally, Mr. Goldwin Smith has truly observed that ``metaphor has been exhausted in depicting the perfection of it, combined with the narrowness of her field;'' and he has justly added that we need not go beyond her own comparison to the art of a miniature painter. To make this latter observation quite exact we must not use the term miniature in its restricted sense, and must think rather of Memling at one end of the history of painting and Meissonier at the other, than of Cosway or any of his kind. And I am not so certain that I should myself use the word ``narrow'' in connection with her. If her world is a microcosm, the cosmic quality of it is at least as eminent as the littleness. She does not touch what she did not feel herself called to paint; I am not so sure that she could not have painted what she did not feel herself called to touch. It is at least remarkable that in two very short periods of writing---one of about three years, and another of not much more than five---she executed six capital works, and has not left a single failure. It is possible that the romantic paste in her composition was defective: we must always remember that hardly anybody born in her decade---that of the eighteenth-century seventies---independently exhibited the full romantic quality. Even Scott required hill and mountain and ballad, even Coleridge metaphysics and German to enable them to chip the classical shell. Miss Austen was an English girl, brought up in a country retirement, at the time when ladies went back into the house if there was a white frost which might pierce their kid shoes, when a sudden cold was the subject of the gravest fears, when their studies, their ways, their conduct were subject to all those fantastic limits and restrictions against which Mary Wollstonecraft protested with better general sense than particular taste or judgment. Miss Austen, too, drew back when the white frost touched her shoes; but I think she would have made a pretty good journey even in a black one.}

\textit{For if her knowledge was not very extended, she knew two things which only genius knows. The one was humanity, and the other was art. On the first head she could not make a mistake; her men, though limited, are true, and her women are, in the old sense, ``absolute.'' As to art, if she has never tried idealism, her realism is real to a degree which makes the false realism of our own day look merely dead-alive. Take almost any Frenchman, except the late M. de Maupassant, and watch him laboriously piling up strokes in the hope of giving a complete impression. You get none; you are lucky if, discarding two-thirds of what he gives, you can shape a real impression out of the rest. But with Miss Austen the myriad, trivial, unforced strokes build up the picture like magic. Nothing is false; nothing is superfluous. When (to take the present book only) Mr. Collins changed his mind from Jane to Elizabeth ``while Mrs. Bennet was stirring the fire'' (and we know} how \textit{Mrs. Bennet would have stirred the fire), when Mr. Darcy ``brought his coffee-cup back} himself,'' \textit{the touch in each case is like that of Swift---``taller by the breadth of my nail''---which impressed the half-reluctant Thackeray with just and outspoken admiration. Indeed, fantastic as it may seem, I should put Miss Austen as near to Swift in some ways, as I have put her to Addison in others.}

\textit{This Swiftian quality appears in the present novel as it appears nowhere else in the character of the immortal, the ineffable Mr. Collins. Mr. Collins is really} great; \textit{far greater than anything Addison ever did, almost great enough for Fielding or for Swift himself. It has been said that no one ever was like him. But in the first place,} he \textit{was like him; he is there---alive, imperishable, more real than hundreds of prime ministers and archbishops, of ``metals, semi-metals, and distinguished philosophers.'' In the second place, it is rash, I think, to conclude that an actual Mr. Collins was impossible or non-existent at the end of the eighteenth century. It is very interesting that we possess, in this same gallery, what may be called a spoiled first draught, or an unsuccessful study of him, in John Dashwood. The formality, the under-breeding, the meanness, are there; but the portrait is only half alive, and is felt to be even a little unnatural. Mr. Collins is perfectly natural, and perfectly alive. In fact, for all the ``miniature,'' there is something gigantic in the way in which a certain side, and more than one, of humanity, and especially eighteenth-century humanity, its Philistinism, its well-meaning but hide-bound morality, its formal pettiness, its grovelling respect for rank, its materialism, its selfishness, receives exhibition. I will not admit that one speech or one action of this inestimable man is incapable of being reconciled with reality, and I should not wonder if many of these words and actions are historically true.}

\textit{But the greatness of Mr. Collins could not have been so satisfactorily exhibited if his creatress had not adjusted so artfully to him the figures of Mr. Bennet and of Lady Catherine de Bourgh. The latter, like Mr. Collins himself, has been charged with exaggeration. There is, perhaps, a very faint shade of colour for the charge; but it seems to me very faint indeed. Even now I do not think that it would be impossible to find persons, especially female persons, not necessarily of noble birth, as overbearing, as self-centred, as neglectful of good manners, as Lady Catherine. A hundred years ago, an earl's daughter, the Lady Powerful (if not exactly Bountiful) of an out-of-the-way country parish, rich, long out of marital authority, and so forth, had opportunities of developing these agreeable characteristics which seldom present themselves now. As for Mr. Bennet, Miss Austen, and Mr. Darcy, and even Miss Elizabeth herself, were, I am inclined to think, rather hard on him for the ``impropriety'' of his conduct. His wife was evidently, and must always have been, a quite irreclaimable fool; and unless he had shot her or himself there was no way out of it for a man of sense and spirit but the ironic. From no other point of view is he open to any reproach, except for an excusable and not unnatural helplessness at the crisis of the elopement, and his utterances are the most acutely delightful in the consciously humorous kind---in the kind that we laugh with, not at---that even Miss Austen has put into the mouth of any of her characters. It is difficult to know whether he is most agreeable when talking to his wife, or when putting Mr. Collins through his paces; but the general sense of the world has probably been right in preferring to the first rank his consolation to the former when she maunders over the entail, ``My dear, do not give way to such gloomy thoughts. Let us hope for better things. Let us flatter ourselves that} I \textit{may be the survivor;'' and his inquiry to his colossal cousin as to the compliments which Mr. Collins has just related as made by himself to Lady Catherine, ``May I ask whether these pleasing attentions proceed from the impulse of the moment, or are the result of previous study?'' These are the things which give Miss Austen's readers the pleasant shocks, the delightful thrills, which are felt by the readers of Swift, of Fielding, and we may here add, of Thackeray, as they are felt by the readers of no other English author of fiction outside of these four.}

\textit{The goodness of the minor characters in} Pride and Prejudice \textit{has been already alluded to, and it makes a detailed dwelling on their beauties difficult in any space, and impossible in this. Mrs. Bennet we have glanced at, and it is not easy to say whether she is more exquisitely amusing or more horribly true. Much the same may be said of Kitty and Lydia; but it is not every author, even of genius, who would have differentiated with such unerring skill the effects of folly and vulgarity of intellect and disposition working upon the common weaknesses of woman at such different ages. With Mary, Miss Austen has taken rather less pains, though she has been even more unkind to her; not merely in the text, but, as we learn from those interesting traditional appendices which Mr. Austen Leigh has given us, in dooming her privately to marry ``one of Mr. Philips's clerks.'' The habits of first copying and then retailing moral sentiments, of playing and singing too long in public, are, no doubt, grievous and criminal; but perhaps poor Mary was rather the scapegoat of the sins of blue stockings in that Fordyce-belectured generation. It is at any rate difficult not to extend to her a share of the respect and affection (affection and respect of a peculiar kind; doubtless), with which one regards Mr. Collins, when she draws the moral of Lydia's fall. I sometimes wish that the exigencies of the story had permitted Miss Austen to unite these personages, and thus at once achieve a notable mating and soothe poor Mrs. Bennet's anguish over the entail.}

\textit{The Bingleys and the Gardiners and the Lucases, Miss Darcy and Miss de Bourgh, Jane, Wickham, and the rest, must pass without special comment, further than the remark that Charlotte Lucas (her egregious papa, though delightful, is just a little on the thither side of the line between comedy and farce) is a wonderfully clever study in drab of one kind, and that Wickham (though something of Miss Austen's hesitation of touch in dealing with young men appears) is a not much less notable sketch in drab of another. Only genius could have made Charlotte what she is, yet not disagreeable; Wickham what he is, without investing him either with a cheap Don Juanish attractiveness or a disgusting rascality. But the hero and the heroine are not tints to be dismissed.}

\textit{Darcy has always seemed to me by far the best and most interesting of Miss Austen's heroes; the only possible competitor being Henry Tilney, whose part is so slight and simple that it hardly enters into comparison. It has sometimes, I believe, been urged that his pride is unnatural at first in its expression and later in its yielding, while his falling in love at all is not extremely probable. Here again I cannot go with the objectors. Darcy's own account of the way in which his pride had been pampered, is perfectly rational and sufficient; and nothing could be, psychologically speaking, a} causa verior \textit{for its sudden restoration to healthy conditions than the shock of Elizabeth's scornful refusal acting on a nature} ex hypothesi \textit{generous. Nothing in even our author is finer and more delicately touched than the change of his demeanour at the sudden meeting in the grounds of Pemberley. Had he been a bad prig or a bad coxcomb, he might have been still smarting under his rejection, or suspicious that the girl had come husband-hunting. His being neither is exactly consistent with the probable feelings of a man spoilt in the common sense, but not really injured in disposition, and thoroughly in love. As for his being in love, Elizabeth has given as just an exposition of the causes of that phenomenon as Darcy has of the conditions of his unregenerate state, only she has of course not counted in what was due to her own personal charm.}

\textit{The secret of that charm many men and not a few women, from Miss Austen herself downwards, have felt, and like most charms it is a thing rather to be felt than to be explained. Elizabeth of course belongs to the} allegro \textit{or} allegra \textit{division of the army of Venus. Miss Austen was always provokingly chary of description in regard to her beauties; and except the fine eyes, and a hint or two that she had at any rate sometimes a bright complexion, and was not very tall, we hear nothing about her looks. But her chief difference from other heroines of the lively type seems to lie first in her being distinctly clever---almost strong-minded, in the better sense of that objectionable word---and secondly in her being entirely destitute of ill-nature for all her propensity to tease and the sharpness of her tongue. Elizabeth can give at least as good as she gets when she is attacked; but she never ``scratches,'' and she never attacks first. Some of the merest obsoletenesses of phrase and manner give one or two of her early speeches a slight pertness, but that is nothing, and when she comes to serious business, as in the great proposal scene with Darcy (which is, as it should be, the climax of the interest of the book), and in the final ladies' battle with Lady Catherine, she is unexceptionable. Then too she is a perfectly natural girl. She does not disguise from herself or anybody that she resents Darcy's first ill-mannered personality with as personal a feeling. (By the way, the reproach that the ill-manners of this speech are overdone is certainly unjust; for things of the same kind, expressed no doubt less stiltedly but more coarsely, might have been heard in more than one ball-room during this very year from persons who ought to have been no worse bred than Darcy.) And she lets the injury done to Jane and the contempt shown to the rest of her family aggravate this resentment in the healthiest way in the world.}

\textit{Still, all this does not explain her charm, which, taking beauty as a common form of all heroines, may perhaps consist in the addition to her playfulness, her wit, her affectionate and natural disposition, of a certain fearlessness very uncommon in heroines of her type and age. Nearly all of them would have been in speechless awe of the magnificent Darcy; nearly all of them would have palpitated and fluttered at the idea of proposals, even naughty ones, from the fascinating Wickham. Elizabeth, with nothing offensive, nothing} viraginous, \textit{nothing of the ``New Woman'' about her, has by nature what the best modern (not ``new'') women have by education and experience, a perfect freedom from the idea that all men may bully her if they choose, and that most will away with her if they can. Though not in the least ``impudent and mannish grown,'' she has no mere sensibility, no nasty niceness about her. The form of passion common and likely to seem natural in Miss Austen's day was so invariably connected with the display of one or the other, or both of these qualities, that she has not made Elizabeth outwardly passionate. But I, at least, have not the slightest doubt that she would have married Darcy just as willingly without Pemberley as with it, and anybody who can read between lines will not find the lovers' conversations in the final chapters so frigid as they might have looked to the Della Cruscans of their own day, and perhaps do look to the Della Cruscans of this.}

\textit{And, after all, what is the good of seeking for the reason of charm?---it is there. There were better sense in the sad mechanic exercise of determining the reason of its absence where it is not. In the novels of the last hundred years there are vast numbers of young ladies with whom it might be a pleasure to fall in love; there are at least five with whom, as it seems to me, no man of taste and spirit can help doing so. Their names are, in chronological order, Elizabeth Bennet, Diana Vernon, Argemone Lavington, Beatrix Esmond, and Barbara Grant. I should have been most in love with Beatrix and Argemone; I should, I think, for mere occasional companionship, have preferred Diana and Barbara. But to live with and to marry, I do not know that any one of the four can come into competition with Elizabeth.}

\textit{GEORGE SAINTSBURY.}

\chapter{Jane Austen}
\section{From \textit{The Common Reader}, by Virgina Woolf}

It is probable that if Miss Cassandra Austen had had her way, we should have had nothing of Jane Austen's except her novels. To her elder sister alone did she write freely; to her alone she confided her hopes and, if rumour is true, the one great disappointment of her life; but when Miss Cassandra Austen grew old, and the growth of her sister's fame made her suspect that a time might come when strangers would pry and scholars speculate, she burnt, at great cost to herself, every letter that could gratify their curiosity, and spared only what she judged too trivial to be of interest.

Hence our knowledge of Jane Austen is derived from a little gossip, a few letters, and her books. As for the gossip, gossip which has survived its day is never despicable; with a little rearrangement it suits our purpose admirably. For example, Jane ``is not at all pretty and very prim, unlike a girl of twelve \dots Jane is whimsical and affected,'' says little Philadelphia Austen of her cousin. Then we have Mrs. Mitford, who knew the Austens as girls and thought Jane ``the prettiest, silliest, most affected, husband-hunting butterfly she ever remembers''. Next, there is Miss Mitford's anonymous friend ``who visits her now [and] says that she has stiffened into the most perpendicular, precise, taciturn piece of 'single blessedness' that ever existed, and that, until \textit{Pride and Prejudice} showed what a precious gem was hidden in that unbending case, she was no more regarded in society than a poker or firescreen\dots . The case is very different now,'' the good lady goes on; ``she is still a poker—but a poker of whom everybody is afraid\dots . A wit, a delineator of character, who does not talk is terrific indeed!'' On the other side, of course, there are the Austens, a race little given to panegyric of themselves, but nevertheless, they say, her brothers ``were very fond and very proud of her. They were attached to her by her talents, her virtues, and her engaging manners, and each loved afterwards to fancy a resemblance in some niece or daughter of his own to the dear sister Jane, whose perfect equal they yet never expected to see.'' Charming but perpendicular, loved at home but feared by strangers, biting of tongue but tender of heart—these contrasts are by no means incompatible, and when we turn to the novels we shall find ourselves stumbling there too over the same complexities in the writer.

To begin with, that prim little girl whom Philadelphia found so unlike a child of twelve, whimsical and affected, was soon to be the authoress of an astonishing and unchildish story, \textit{Love and Friendship}, which, incredible though it appears, was written at the age of fifteen. It was written, apparently, to amuse the schoolroom; one of the stories in the same book is dedicated with mock solemnity to her brother; another is neatly illustrated with water-colour heads by her sister. There are jokes which, one feels, were family property; thrusts of satire, which went home because all little Austens made mock in common of fine ladies who ``sighed and fainted on the sofa''.

Brothers and sisters must have laughed when Jane read out loud her last hit at the vices which they all abhorred. ``I die a martyr to my grief for the loss of Augustius. One fatal swoon has cost me my life. Beware of Swoons, Dear Laura\dots . Run mad as often as you chuse, but do not faint\dots .'' And on she rushed, as fast as she could write and quicker than she could spell, to tell the incredible adventures of Laura and Sophia, of Philander and Gustavus, of the gentleman who drove a coach between Edinburgh and Stirling every other day, of the theft of the fortune that was kept in the table drawer, of the starving mothers and the sons who acted Macbeth. Undoubtedly, the story must have roused the schoolroom to uproarious laughter. And yet, nothing is more obvious than that this girl of fifteen, sitting in her private corner of the common parlour, was writing not to draw a laugh from brother and sisters, and not for home consumption. She was writing for everybody, for nobody, for our age, for her own; in other words, even at that early age Jane Austen was writing. One hears it in the rhythm and shapeliness and severity of the sentences. ``She was nothing more than a mere good tempered, civil, and obliging young woman; as such we could scarcely dislike her—she was only an object of contempt.'' Such a sentence is meant to outlast the Christmas holidays. Spirited, easy, full of fun, verging with freedom upon sheer nonsense,—\textit{Love and Friendship} is all that, but what is this note which never merges in the rest, which sounds distinctly and penetratingly all through the volume? It is the sound of laughter. The girl of fifteen is laughing, in her corner, at the world.

Girls of fifteen are always laughing. They laugh when Mr. Binney helps himself to salt instead of sugar. They almost die of laughing when old Mrs. Tomkins sits down upon the cat. But they are crying the moment after. They have no fixed abode from which they see that there is something eternally laughable in human nature, some quality in men and women that for ever excites our satire. They do not know that Lady Greville who snubs, and poor Maria who is snubbed, are permanent features of every ball-room. But Jane Austen knew it from her birth upwards. One of those fairies who perch upon cradles must have taken her a flight through the world directly she was born. When she was laid in the cradle again she knew not only what the world looked like, but had already chosen her kingdom. She had agreed that if she might rule over that territory, she would covet no other. Thus at fifteen she had few illusions about other people and none about herself. Whatever she writes is finished and turned and set in its relation, not to the parsonage, but to the universe. She is impersonal; she is inscrutable. When the writer, Jane Austen, wrote down in the most remarkable sketch in the book a little of Lady Greville's conversation, there is no trace of anger at the snub which the clergyman's daughter, Jane Austen, once received. Her gaze passes straight to the mark, and we know precisely where, upon the map of human nature, that mark is. We know because Jane Austen kept to her compact; she never trespassed beyond her boundaries. Never, even at the emotional age of fifteen, did she round upon herself in shame, obliterate a sarcasm in a spasm of compassion, or blur an outline in a mist of rhapsody. Spasms and rhapsodies, she seems to have said, pointing with her stick, end \textit{there}; and the boundary line is perfectly distinct. But she does not deny that moons and mountains and castles exist—on the other side. She has even one romance of her own. It is for the Queen of Scots. She really admired her very much. ``One of the first characters in the world,'' she called her, ``a bewitching Princess whose only friend was then the Duke of Norfolk, and whose only ones now Mr. Whitaker, Mrs. Lefroy, Mrs. Knight and myself.'' With these words her passion is neatly circumscribed, and rounded with a laugh. It is amusing to remember in what terms the young Brontës wrote, not very much later, in their northern parsonage, about the Duke of Wellington.

The prim little girl grew up. She became ``the prettiest, silliest, most affected husband-hunting butterfly'' Mrs. Mitford ever remembered, and, incidentally, the authoress of a novel called \textit{Pride and Prejudice}, which, written stealthily under cover of a creaking door, lay for many years unpublished. A little later, it is thought, she began another story, \textit{The Watsons}, and being for some reason dissatisfied with it, left it unfinished. Unfinished and unsuccessful, it may throw more light upon its writer's genius than the polished masterpiece blazing in universal fame. Her difficulties are more apparent in it, and the method she took to overcome them less artfully concealed. To begin with, the stiffness and the bareness of the first chapters prove that she was one of those writers who lay their facts out rather baldly in the first version and then go back and back and back and cover them with flesh and atmosphere. How it would have been done we cannot say—by what suppressions and insertions and artful devices. But the miracle would have been accomplished; the dull history of fourteen years of family life would have been converted into another of those exquisite and apparently effortless introductions; and we should never have guessed what pages of preliminary drudgery Jane Austen forced her pen to go through. Here we perceive that she was no conjuror after all. Like other writers, she had to create the atmosphere in which her own peculiar genius could bear fruit. Here she fumbles; here she keeps us waiting. Suddenly, she has done it; now things can happen as she likes things to happen. The Edwards' are going to the ball. The Tomlinsons' carriage is passing; she can tell us that Charles is ``being provided with his gloves and told to keep them on''; Tom Musgrove retreats to a remote corner with a barrel of oysters and is famously snug. Her genius is freed and active. At once our senses quicken; we are possessed with the peculiar intensity which she alone can impart. But of what is it all composed? Of a ball in a country town; a few couples meeting and taking hands in an assembly room; a little eating and drinking; and for catastrophe, a boy being snubbed by one young lady and kindly treated by another. There is no tragedy and no heroism. Yet for some reason the little scene is moving out of all proportion to its surface solemnity. We have been made to see that if Emma acted so in the ball-room, how considerate, how tender, inspired by what sincerity of feeling she would have shown herself in those graver crises of life which, as we watch her, come inevitably before our eyes. Jane Austen is thus a mistress of much deeper emotion than appears upon the surface. She stimulates us to supply what is not there. What she offers is, apparently, a trifle, yet is composed of something that expands in the reader's mind and endows with the most enduring form of life scenes which are outwardly trivial. Always the stress is laid upon character. How, we are made to wonder, will Emma behave when Lord Osborne and Tom Musgrove make their call at five minutes before three, just as Mary is bringing in the tray and the knife-case? It is an extremely awkward situation. The young men are accustomed to much greater refinement. Emma may prove herself ill-bred, vulgar, a nonentity. The turns and twists of the dialogue keep us on the tenterhooks of suspense. Our attention is half upon the present moment, half upon the future. And when, in the end, Emma behaves in such a way as to vindicate our highest hopes of her, we are moved as if we had been made witnesses of a matter of the highest importance. Here, indeed, in this unfinished and in the main inferior story are all the elements of Jane Austen's greatness. It has the permanent quality of literature. Think away the surface animation, the likeness to life, and there remains to provide a deeper pleasure, an exquisite discrimination of human values. Dismiss this too from the mind and one can dwell with extreme satisfaction upon the more abstract art which, in the ball-room scene, so varies the emotions and proportions the parts that it is possible to enjoy it, as one enjoys poetry, for itself, and not as a link which carries the story this way and that.

But the gossip says of Jane Austen that she was perpendicular, precise, and taciturn—``a poker of whom everybody is afraid''. Of this too there are traces; she could be merciless enough; she is one of the most consistent satirists in the whole of literature. Those first angular chapters of \textit{The Watsons} prove that hers was not a prolific genius; she had not, like Emily Brontë, merely to open the door to make herself felt. Humbly and gaily she collected the twigs and straws out of which the nest was to be made and placed them neatly together. The twigs and straws were a little dry and a little dusty in themselves. There was the big house and the little house; a tea party, a dinner party, and an occasional picnic; life was hedged in by valuable connections and adequate incomes; by muddy roads, wet feet, and a tendency on the part of the ladies to get tired; a little money supported it, a little consequence, and the education commonly enjoyed by upper middle-class families living in the country. Vice, adventure, passion were left outside. But of all this prosiness, of all this littleness, she evades nothing, and nothing is slurred over. Patiently and precisely she tells us how they ``made no stop anywhere till they reached Newbury, where a comfortable meal, uniting dinner and supper, wound up the enjoyments and fatigues of the day''. Nor does she pay to conventions merely the tribute of lip homage; she believes in them besides accepting them. When she is describing a clergyman, like Edmund Bertram, or a sailor, in particular, she appears debarred by the sanctity of his office from the free use of her chief tool, the comic genius, and is apt therefore to lapse into decorous panegyric or matter-of-fact description. But these are exceptions; for the most part her attitude recalls the anonymous ladies' ejaculation—``A wit, a delineator of character, who does not talk is terrific indeed!'' She wishes neither to reform nor to annihilate; she is silent; and that is terrific indeed. One after another she creates her fools, her prigs, her worldlings, her Mr. Collins', her Sir Walter Elliotts, her Mrs. Bennetts. She encircles them with the lash of a whip-like phrase which, as it runs round them, cuts out their silhouettes for ever. But there they remain; no excuse is found for them and no mercy shown them. Nothing remains of Julia and Maria Bertram when she has done with them; Lady Bertram is left ``sitting and calling to Pug and trying to keep him from the flower beds'' eternally. A divine justice is meted out; Dr. Grant, who begins by liking his goose tender, ends by bringing on ``apoplexy and death, by three great institutionary dinners in one week''. Sometimes it seems as if her creatures were born merely to give Jane Austen the supreme delight of slicing their heads off. She is satisfied; she is content; she would not alter a hair on anybody's head, or move one brick or one blade of grass in a world which provides her with such exquisite delight.

Nor, indeed, would we. For even if the pangs of outraged vanity, or the heat of moral wrath, urged us to improve away a world so full of spite, pettiness, and folly, the task is beyond our powers. People are like that—the girl of fifteen knew it; the mature woman proves it. At this very moment some Lady Bertram finds it almost too trying to keep Pug from the flower beds; she sends Chapman to help Miss Fanny, a little late. The discrimination is so perfect, the satire so just that, consistent though it is, it almost escapes our notice. No touch of pettiness, no hint of spite, rouses us from our contemplation. Delight strangely mingles with our amusement. Beauty illumines these fools.

That elusive quality is indeed often made up of very different parts, which it needs a peculiar genius to bring together. The wit of Jane Austen has for partner the perfection of her taste. Her fool is a fool, her snob is a snob, because he departs from the model of sanity and sense which she has in mind, and conveys to us unmistakably even while she makes us laugh. Never did any novelist make more use of an impeccable sense of human values. It is against the disc of an unerring heart, an unfailing good taste, an almost stern morality, that she shows up those deviations from kindness, truth, and sincerity which are among the most delightful things in English literature. She depicts a Mary Crawford in her mixture of good and bad entirely by this means. She lets her rattle on against the clergy, or in favour of a baronetage and ten thousand a year with all the ease and spirit possible; but now and again she strikes one note of her own, very quietly, but in perfect tune, and at once all Mary Crawford's chatter, though it continues to amuse, rings flat. Hence the depth, the beauty, the complexity of her scenes. From such contrasts there comes a beauty, a solemnity even which are not only as remarkable as her wit, but an inseparable part of it. In \textit{The Watsons} she gives us a foretaste of this power; she makes us wonder why an ordinary act of kindness, as she describes it, becomes so full of meaning. In her masterpieces, the same gift is brought to perfection. Here is nothing out of the way; it is midday in Northamptonshire; a dull young man is talking to rather a weakly young woman on the stairs as they go up to dress for dinner, with housemaids passing. But, from triviality, from commonplace, their words become suddenly full of meaning, and the moment for both one of the most memorable in their lives. It fills itself; it shines; it glows; it hangs before us, deep, trembling, serene for a second; next, the housemaid passes, and this drop in which all the happiness of life has collected gently subsides again to become part of the ebb and flow of ordinary existence.

What more natural then, with this insight into their profundity, than that Jane Austen should have chosen to write of the trivialities of day to day existence, of parties, picnics, and country dances? No ``suggestions to alter her style of writing'' from the Prince Regent or Mr. Clarke could tempt her; no romance, no adventure, no politics or intrigue could hold a candle to life on a country house staircase as she saw it. Indeed, the Prince Regent and his librarian had run their heads against a very formidable obstacle; they were trying to tamper with an incorruptible conscience, to disturb an infallible discretion. The child who formed her sentences so finely when she was fifteen never ceased to form them, and never wrote for the Prince Regent or his Librarian, but for the world at large. She knew exactly what her powers were, and what material they were fitted to deal with as material should be dealt with by a writer, whose standard of finality was high. There were impressions that lay outside her province; emotions that by no stretch or artifice could be properly coated and covered by her own resources. For example, she could not make a girl talk enthusiastically of banners and chapels. She could not throw herself wholeheartedly into a romantic moment. She had all sorts of devices for evading scenes of passion. Nature and its beauties she approached in a sidelong way of her own. She describes a beautiful night without once mentioning the moon. Nevertheless, as we read the few formal phrases about ``the brilliancy of an unclouded night and the contrast of the deep shade of the woods'' the night is at once as ``solemn, and soothing, and lovely'' as she tells us, quite simply, that it was.

The balance of her gifts was singularly perfect. Among her finished novels there are no failures, and among her many chapters few that sink markedly below the level of the others. But, after all, she died at the age of forty-two. She died at the height of her powers. She was still subject to those changes which often make the final period of a writer's career the most interesting of all. Vivacious, irrepressible, gifted with an invention of great vitality, there can be no doubt that she would have written more, had she lived, and it is tempting to consider whether she would not have written differently. The boundaries were marked; moons, mountains, and castles lay on the other side. But was she not sometimes tempted to trespass for a minute? Was she not beginning, in her own gay and brilliant manner, to contemplate a little voyage of discovery?

Let us take \textit{Persuasion}, the last completed novel, and look by its light at the books she might have written had she lived. There is a peculiar beauty and a peculiar dullness in \textit{Persuasion}. The dullness is that which so often marks the transition stage between two different periods. The writer is a little bored. She has grown too familiar with the ways of her world; she no longer notes them freshly. There is an asperity in her comedy which suggests that she has almost ceased to be amused by the vanities of a Sir Walter or the snobbery of a Miss Elliott. The satire is harsh, and the comedy crude. She is no longer so freshly aware of the amusements of daily life. Her mind is not altogether on her object. But, while we feel that Jane Austen has done this before, and done it better, we also feel that she is trying to do something which she has never yet attempted. There is a new element in \textit{Persuasion}, the quality, perhaps, that made Dr. Whewell fire up and insist that it was ``the most beautiful of her works''. She is beginning to discover that the world is larger, more mysterious, and more romantic than she had supposed. We feel it to be true of herself when she says of Anne: ``She had been forced into prudence in her youth, she learned romance as she grew older—the natural sequel of an unnatural beginning''. She dwells frequently upon the beauty and the melancholy of nature, upon the autumn where she had been wont to dwell upon the spring. She talks of the ``influence so sweet and so sad of autumnal months in the country''. She marks ``the tawny leaves and withered hedges''. ``One does not love a place the less because one has suffered in it'', she observes. But it is not only in a new sensibility to nature that we detect the change. Her attitude to life itself is altered. She is seeing it, for the greater part of the book, through the eyes of a woman who, unhappy herself, has a special sympathy for the happiness and unhappiness of others, which, until the very end, she is forced to comment upon in silence. Therefore the observation is less of facts and more of feelings than is usual. There is an expressed emotion in the scene at the concert and in the famous talk about woman's constancy which proves not merely the biographical fact that Jane Austen had loved, but the æsthetic fact that she was no longer afraid to say so. Experience, when it was of a serious kind, had to sink very deep, and to be thoroughly disinfected by the passage of time, before she allowed herself to deal with it in fiction. But now, in 1817, she was ready. Outwardly, too, in her circumstances, a change was imminent. Her fame had grown very slowly. ``I doubt'', wrote Mr. Austen Leigh, ``whether it would be possible to mention any other author of note whose personal obscurity was so complete.'' Had she lived a few more years only, all that would have been altered. She would have stayed in London, dined out, lunched out, met famous people, made new friends, read, travelled, and carried back to the quiet country cottage a hoard of observations to feast upon at leisure.

And what effect would all this have had upon the six novels that Jane Austen did not write? She would not have written of crime, of passion, or of adventure. She would not have been rushed by the importunity of publishers or the flattery of friends into slovenliness or insincerity. But she would have known more. Her sense of security would have been shaken. Her comedy would have suffered. She would have trusted less (this is already perceptible in \textit{Persuasion}) to dialogue and more to reflection to give us a knowledge of her characters. Those marvellous little speeches which sum up, in a few minutes' chatter, all that we need in order to know an Admiral Croft or a Mrs. Musgrove for ever, that shorthand, hit-or-miss method which contains chapters of analysis and psychology, would have become too crude to hold all that she now perceived of the complexity of human nature. She would have devised a method, clear and composed as ever, but deeper and more suggestive, for conveying not only what people say, but what they leave unsaid; not only what they are, but what life is. She would have stood farther away from her characters, and seen them more as a group, less as individuals. Her satire, while it played less incessantly, would have been more stringent and severe. She would have been the forerunner of Henry James and of Proust—but enough. Vain are these speculations: the most perfect artist among women, the writer whose books are immortal, died ``just as she was beginning to feel confidence in her own success''.



ALSO INCLUDE: TRILLING
